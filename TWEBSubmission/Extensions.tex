\documentclass[11pt]{article}
\include{rmacros}
%\usepackage{fancyhdr}
\usepackage{enumerate}
%\usepackage{hyperref}
\usepackage{graphicx}
\usepackage{amsmath}
\usepackage{amssymb}
\usepackage{chngpage}
\usepackage[colorlinks]{hyperref}
\setlength{\parindent}{0pt}
\setlength{\parskip}{4pt}

\oddsidemargin = 0.2in
\textwidth = 6.5 in
\textheight = 9.8 in
\headsep = -1in

% \pagestyle{fancy}
% \lhead{\Large \parbox{11cm}{Name} }
% \renewcommand{\headheight}{24pt}

\title{Differences between the journal submission and the conference publication}
\author{Yi Qian and Sibel Adal{\i}}
\date{}
\begin{document}
\maketitle

The following submission provides a significantly expanded version of
the conference paper titled: ``Extended Structural Balance Theory for
Modeling Trust in Social Networks'' which is accepted to appear in PST
2013 Conference on Privacy, Security and Trust.

The theory underlying both papers is the same. However, the journal
submission provides an extensive rewriting of the experimental results
reported in the conference submission, illustrating various properties
of the proposed method.

\begin{itemize}
\item We introduce a number of small stylized graphs and show the
  results of convergence for these graphs. We discuss how our theory
  captures frequently hypothesized properties of trust with the help
  of these graphs.
\item In the conference publication, we have only provided limited set
  of results using Stress Majorization (SM) which provides an exact
  solution to the problem. However, due to the underlying complexity
  of this algorithm, we were only able to test it on small samples of
  the underlying graphs studied. In this paper, we introduce a new
  version of Stress Majorization for sparse graphs (SM/SG) and use it
  to provide many new results that use the full network data. The
  following describes these new methods.
\item We first show that SM and SM/SG have similar performance on the
  same sampled graphs. Then, we use SM/SG to test many different
  properties of our algorithm that were not illustrated in the
  conference publication. In particular, we show how the choice of the
  underlying graph, the choice of neutral edges impacts the result. We
  also illustrate how the performance of the algorithm is impacted by
  removal of edges from the graph. We show to which degree our method
  can predict the sign of edges that will be created in the future. We
  discuss the sensitivity to parameter choices. 
\item In addition, we add two studies to this paper that are crucial
  for illustrating the utility of the proposed method. The first one
  is the study of strong and weak edges. We introduce bi-directional
  edges with the same sign as strong and single directional edges as
  weak edges. We define neutral edges as those with conflicting
  signs. Using this graph, we show that our method is able to achieve
  superior performance for strong edges while providing similar
  performance for weak edges as before.
\item The second crucial study involves justification of the method
  with external validation. We incorporate a study of ratings that is
  not used in the algorithm to illustrate the notion of
  convergence. We show how edges that change sign according to our
  theory also tend to change their ratings more than average. We show
  that the performance remains similar even when we consider strong
  and weak edges. We discuss how network structure can predict this
  change.
\item All the results are discussed in great detail that was not
  possible due to the limited scope of the conference paper. We expand
  introduction and conclusion accordingly to further discuss the new
  findings.
\end{itemize}



\end{document}

